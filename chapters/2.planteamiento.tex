\chapter{Planteamiento del Problema}
\thispagestyle{fancy}
\noindent En el contexto académico el principal objetivo que se tiene es obtener nuevo conocimiento y hacer uso o practica del conocimiento, tradicionalmente en el contexto académico se hace uso de herramientas o medios físicos como: notas generadas a mano, anotaciones en textos o libros físicos o notas dispersas generadas en cuaderno u hojas. Todo esto con el fin de enriquecer nuestro aprendizaje y dar paso a una correcta aplicación del conocimiento.\\

Al analizar esta situación encontramos que puede volverse un problema en el momento en que generemos una gran volumen de notas sobre varios temas de diferentes tópicos y al momento de necesitar consultarlas o mejorarlas no dispongamos de un acceso fácil ya que en su mayoría dependemos de un medio físico para acceder a ellas de igual manera poder tener un acceso fácil y rápido depende del volumen físico del medio físico que se quiera consultar y del lugar en donde necesitemos o dispongamos de su acceso.\\

Sin embargo la tecnología a mejorado y nos ha permitido tener beneficios como: un mejor acceso a la información, resguardar información que necesitemos, brindarnos un fácil acceso a esta y no depender totalmente de un medio físico para poder asegurarnos el acceso a la información. Ahora si bien existen herramientas modernas como: Notion, Obsidian, Google Docs, entre otras, que facilitan obtener los beneficios ya mencionados además revolucionan la manera en la podemos generar nuestras notas y apuntes de manera digital.\\

Estas herramientas no nos garantizan la seguridad de los datos que generamos al usarlas así como su uso que le dan a nuestros datos, de igual manera no nos aseguran el respaldo de los datos que almacenemos en estas herramientas, se genera una incertidumbre de que pasara con los datos si estas herramientas llegan a desaparecer.\\

Por ultimo uno de los principales problemas a los que se enfrentan los estudiantes y profesionales es la organización de las notas que generen a lo largo de su vida, teniendo como resultado un sesgo en cuanto se llega a conocer sobre un tema o las metodologías con las que pueden aprender generando un sentimiento de incertidumbre y miedo.\\

Por lo tanto el desarrollo de una aplicación en la cual los usuarios puedan confiar, asegurar un correcto uso de la información, garantizar un fácil acceso a la información y  puedan consultar sin la necesidad de disponer del medio puede ser  de gran utilidad para los usuarios.\\

Si además esta aplicación pueda dar recomendaciones acerca de su modelo de aprendizaje y ayudar a generar notas de calidad sin depender totalmente de medios físicos o plataformas que en algún momento privatizaran todo lo mencionado será de gran valor para los usuarios.\\


\section{Preguntas de investigación}
\noindent A partir del planteamiento del problema se derivan las siguientes preguntas de investigación que guiarán el desarrollo del proyecto:
\begin{itemize}[label={}]
\item ¿Cómo la consulta de notas puede ayudar al desarrollo y mejora del conocimiento?
\item ¿Cuál es el beneficio de generar notas para los usuarios?
\item ¿El desarrollo de un sistema offline para la consulta de las notas es posible sin la total dependencia de una conexión?
\item ¿Cómo se puede hacer colaboración entre usuarios sin la total dependencia de una conexión local o remota?
\end{itemize}

\section{Hipótesis}

\noindent La creación de una aplicación de gestión de notas inteligente que permita a los usuarios crear, organizar y consultar una base de conocimientos personalizada sin depender de una conexión a internet, garantizando la privacidad y seguridad de sus datos, mejorará significativamente la accesibilidad y utilidad de las notas académicas para estudiantes y profesionales.

\section{Objetivo general}

Crear una aplicación inteligente de gestión de notas para realizar consultas con lenguaje natural empleando procesamiento local de datos y técnicas de inteligencia artificial.



\section{Objetivos específicos}

Para lograr el desarrollo del gestor de notas inteligente se divide el proyecto en tres etapas: Proyecto Terminal 1, Proyecto Terminal 2 y Proyecto Terminal 3, estas etapas tienen los siguientes objetivos específicos:

\subsection*{Proyecto Terminal 1}

\begin{enumerate}
\item Analizar la problemática de la gestión de notas para definir los requerimientos y el alcance funcional de la aplicación \emph{Gestor inteligente para notas académicas}.

\item Delimitar las funciones principales que deberá cumplir la aplicación al finalizar su desarrollo, estableciendo criterios medibles.

\item Diseñar la arquitectura general del sistema, representando flujos de trabajo mediante diagramas UML que muestren la interacción entre los componentes.

\item Elaborar el modelo de datos que permitirá almacenar y gestionar la información necesaria para el funcionamiento del sistema.

\item Determinar las tecnologías, frameworks y herramientas más adecuadas para el desarrollo de la aplicación.
\end{enumerate}

\subsection*{Proyecto Terminal 2}

\begin{enumerate}

\item Desarrollar un prototipo funcional del \emph{Gestor inteligente para notas académicas} que incluya las principales funcionalidades orientadas a la interacción con el usuario.

\item Implementar una interfaz de usuario intuitiva que permita una experiencia de uso amigable y reduzca la posibilidad de errores.

\item Construir una base de conocimiento utilizando bases de datos vectoriales (como FAISS o ChromaDB) para almacenar y gestionar información en distintos formatos (texto, imágenes, audio).

\item Implementar un motor de búsqueda semántica basado en \textit{embeddings} y modelos de lenguaje (como Llama 3 o \textit{Mistral}) para optimizar la recuperación de información relevante.

\end{enumerate}

\subsection*{Proyecto Terminal 3}

\begin{enumerate}

\item Completar el desarrollo del prototipo mediante la implementación de un sistema de replicación como \textit{peer-to-peer} (\emph{P2P}) que permita la sincronización y colaboración entre bases de conocimiento de diferentes usuarios.

\item Integrar funciones de reconocimiento de voz (\emph{STT}) y síntesis de voz (\emph{TTS}) que amplíen las modalidades de entrada y salida de información dentro de la aplicación.

\item Evaluar el desempeño, la seguridad y la estabilidad de la aplicación mediante pruebas finales que aseguren el cumplimiento de los requisitos definidos en fases anteriores.

\item Documentar los resultados, reportar defectos encontrados y analizarlos para establecer mejoras y asi poder validar el cumplimiento de los objetivos generales del proyecto.

\end{enumerate}


\section{Justificación}

Esta aplicación permitirá a los usuarios crear, organizar y consultar una base de conocimientos personalizada sin depender de una conexión a internet, garantizando la privacidad y seguridad de sus datos. La aplicación integrará tecnologías avanzadas como bases de datos vectoriales para búsquedas semánticas, modelos de lenguaje locales para interacción mediante chatbots, y sistemas distribuidos para facilitar la colaboración entre usuarios.\\

El gestor de notas inteligentes tendrá importancia porque proporcionará una solución a la necesidad de organizar las notas y estructurar el conocimiento de manera personalizada, dando la confianza de que la información sera resguardada de forma segura y privada, asi mismo, el usuario tendrá acceso para consultar, de manera personalizada y eficiente, las notas que genere. También, será de utilidad para docentes, profesionales o personas interesadas en consultar o compartir notas en cualquier momento.\\

A continuación se presentan los siguientes aspectos que evidencian la relevancia y el valor de esta propuesta: \\

\begin{description}
\item[La privacidad del contenido e información:] al operar de manera local se garantiza tener un mayor control sobre los datos del usuario, minimizando el riesgo de  filtraciones no deseadas. De igual manera, al descartar el uso de hardware externo, se optimiza el rendimiento del procesamiento de la información, evitando consultas reiteradas de fuentes externas.

\item[Accesibilidad ampliada:] al no requerir conexión a internet, el acceso a la información puede realizarse desde cualquier ubicación geográfica, garantizando la disponibilidad constante para el usuario.

\item[Colaboración entre usuarios:] La ausencia de una arquitectura centralizada facilita el intercambio y la complementación del conocimiento, especialmente entre la comunidad estudiantil.

\item[Lenguaje natural y procesamiento de información:] La implementación del procesamiento local del lenguaje natural y sistemas distribuidos posiciona a este proyecto en áreas de gran potencial de desarrollo y aplicación en el ámbito educativo.

\item[Mejora en la experiencia del usuario:] El empleo de procesamiento de información multimodal elimina la dependencia de un único método para alimentar la base de conocimientos y el acceso a ésta, facilitando el manejo sin complicaciones para el usuario.

\item[Acceso multiplataforma:]  Al hacer uso de tecnologías multiplataforma, podemos asegurar que el gestor de notas inteligente este disponible en diversos dispositivos y sistemas operativos, evitando así la limitación a una única plataforma.

\end{description}




\section{Alcances del proyecto}
El alcance que se plantea para la aplicación gestor de notas inteligentes es desarrollar una aplicación multiplataforma en la que los usuarios tendrán acceso a las siguientes funcionalidades.\\

El usuario podrá generar una base de conocimiento con diferentes contenidos digitales, esta funcionalidad se logrará con herramientas de bases de datos vectoriales que permitirán lograr una búsqueda semántica entre los datos que el usuario ingrese además estas herramientas permiten una escalabilidad en la de información que la base de conocimiento del usuario posea esto permitirá y garantizará una búsqueda optimizada, delimitada y con términos sugeridos según el usuario lo requiera de igual forma al garantizar esta implementación permitirá una integración con el Chat Bot.\\

Como segundo punto a desarrollar es la implementación de un sistema distribuido con mecanismos como peer-to-peer (P2P), (libp2p, WebRTC en LAN), estos permitirán prever errores al no prescindir de un sistema centralizado permitirá utilizar a diferentes usuarios como un respaldo asimismo podremos sincronizar y compartir múltiples nodos para compartir bases de conocimiento de otros usuarios, estas funcionalidades convierten a la aplicación gestor de notas inteligente en una aplicación mas robusta también garantizaremos la colaboración y el enriquecimiento del conocimiento con otros usuarios.\\

Por ultimo el proyecto se enfocara en el desarrollo y la implementación de un Chat Bot, mediante modelos de lenguaje local (Llama 3, Mistral o similares), permitiendo a los usuarios una interacción cercana y privada con el contenido de su base de conocimientos generado esto también representa un punto de innovación en el desarrollo del proyecto.\\

\section{Limitaciones}
\noindent A pesar de la utilidad que este proyecto tiene, podemos describir limitaciones tecnológicas y de recursos que pueden dificultar su desarrollo que se describiremos a continuación:\\

El proyecto no hará uso de conexión a internet por lo que la replicación o colaboración entre usuarios puede verse limitada por la conexión y uso de redes locales, esto representa una limitación en cuanto a la cantidad de usuarios y la colaboración entre los mismos.\\

Otro punto a destacar es el uso de hardware del usuario, ya que este limita el rendimiento de la aplicación al incluir funcionalidades como el procesamiento de datos esto requiere un consumo de recursos de hardware que se esté empleando.\\

Una limitación metodológica que pude encontrar, es en el material con el que se generara la base de conocimientos del usuario, la certidumbre que la información del usuario tenga ya que si este no corroborará o cerciorara que la información sea fiable podemos caer en una falsa \textit{praxis} de la información, con la que se este generando una nota o conversación.\\
\\


Cabe destacar que el diseño de una interfaz de usuario y una experiencia de usuario pueden limitar el desarrollo de la aplicación ya que esto tiene como principal objetivo el generar atractivo y confianza al usuario, garantizando que su información este segura y se mantendrá privada de consultas externas.\\

Por ultimo otra limitante que se encontró en cuanto a los recursos en el desarrollo de la aplicación de gestión de notas es el tiempo que se tiene para llegar a cumplir los objetivos deseados, ya que al disponer de solo tres trimestres, el tiempo que se tiene es relativamente corto.\\
