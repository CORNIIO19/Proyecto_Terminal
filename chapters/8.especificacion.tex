\chapter{Especificación Técnica}
\thispagestyle{fancy}

Para el desarrollo de la aplicacion de gestión de notas, se seleccionó un stack tecnológico que garantiza privacidad, procesamiento local y colaboración sin conexión a internet.
Se propone hacer uso de las tecnologias JavaScript y TypeScript para la construcción de la aplicación, con posibilidad de ampliar a dispositivos móviles usando React Native o Flutter. Para el procesamiento de lenguaje natural y manejo de modelos locales se propone hacer uso de modelos como Llama 3 o Mistral.
La base de datos principal será una solución vectorial como FAISS o ChromaDB, permitiendo búsquedas semánticas eficientes, complementada con bases ligeras locales como SQL.
La sincronización entre usuarios se realizará mediante redes peer-to-peer, usando libp2p y WebRTC para comunicaciones seguras y sin servidores centrales.
Para la interacción multimodal, se integrarán tecnologías de reconocimiento y síntesis de voz locales, como DeepSpeech o Whisper para STT, y Festival TTS para Text-to-Speech.
La interfaz se desarrollará con frameworks modernos como React o Angular, buscando una experiencia accesible y confiable.
Este conjunto tecnológico asegura una solución robusta, privada y colaborativa, alineada con los objetivos del proyecto.