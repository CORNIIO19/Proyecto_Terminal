\chapter{Estado del Arte}
En esta seccion realizaré un analisis comparativo con las principales soluciones de software disponibles en el mercado y que tienen relacion con el proyecto gestor de notas inteligente, este analisis nos permitira conocer los principales diferenciadores con respecto a estas soluciones permitiendonos ademas mostrar la importancia o los beneficios del desarrollo de este proyecto, para esto tomaré aplicaciones como: Notion, Obsidian, NotebookLM .\\

\section{Notion} Notion es una herramienta multiplataforma de trabajo todo en uno, esta herramienta se centra principalmente en generar espacios de trabajo propios o para equipos, estos espacios pueden ser totalmente personalizables ademas de facilitar la creacion de documentos, planificacion del trabajo, el seguimiento y gestion de proyectos, Notion permite la creaion de bases de conocimientos y la estructura de notas.\cite{Notion_2025,Notion_Personal} \\

\section{Obsidian} Obsidian es una aplicacion gratuita y flexible para organizar conocimiento, desde llevar la creacion de diarios, hasta crear bases de conocimientos y gestionar proyectos, obsidian crea los archivos en formatos abiertos por lo que no se depende de un solo formato ademas estos archivos los guarda de manera privada. \cite{Obsidian}\\

\section{NotebookLM} NotebookLM se define como un compañero o ayudante de investigacion y reflexion que se basa unicamente en informacion en la que confie el usuario, esta aplicacion incorpora IA usando modelos Gemini, NotebookLM ofrece la posibilidad de potenciar el estudio, organizar ideas y generar nuevas ideas a travez de consultas y la generacion de diversos contenidos. \cite{NotebookLM}\\

\section{Comparativa}
\begin{table}[H]
\centering
\scriptsize
\caption{Comparativa entre Gestor de Notas Inteligente y aplicaciones comerciales}\label{tab:comparativa_final}
\renewcommand{\arraystretch}{0.9}
\begin{tabularx}{\textwidth}{p{2.3cm} X X X X}
\toprule
\textbf{Categoría} & \textbf{Notion} & \textbf{Obsidian} & \textbf{NotebookLM} & \textbf{Gestor de Notas Inteligente} \\
\midrule
Enfoque & Productividad y colaboración. & Conocimiento personal (Markdown local). & Asistente de IA para lectura y resumen. & IA local para gestión y consulta de notas. \\
\addlinespace
Almacenamiento & En la nube. & Local, con opción de sincronizar. & En la nube (Google). & 100\% local y privada. \\
\addlinespace
Privacidad & Datos gestionados por servidores externos. & Control parcial del usuario. & Total dependencia de Google. & Privacidad y control total del usuario. \\
\addlinespace
Acceso offline & Limitado. & Parcial. & No disponible. & Totalmente funcional sin conexión. \\
\addlinespace
IA & Basada en OpenAI (nube). & Sin IA nativa (plugins externos). & IA de Google (nube). & Modelos locales. \\
\addlinespace
Colaboración & En tiempo real (nube). & Limitada o manual. & Sin colaboración directa. & distribuido. \\
\addlinespace
Experiencia de usuario & Intuitiva pero con curva alta. & Técnica y avanzada. & Simple y guiada. & Intuitiva, ligera y accesible. \\
\addlinespace
Educativo & General y flexible. & Enfocado a investigación personal. & Lectura y síntesis. & Dirigido a estudiantes y docentes. \\
\addlinespace
Plataformas & Web, escritorio y móvil. & Escritorio y móvil. & Web y Movil. & Web, escritorio y móvil. \\
\addlinespace
Filosofía & Productividad visual y colaborativa. & Conexión entre ideas. & Automatización mediante IA. & Autonomía, privacidad y conocimiento local. \\
\bottomrule
\end{tabularx}
\end{table}

\noindent Por lo tanto esta comparativa mostrada en el cuadro \ref{tab:comparativa_final} nos muestra las principales diferencias entre las aplicaciones existente en el mercado y el gestor de notas inteligente en el cual podemos concluir que brindar seguridad a los usuarios de que la informacion estara siempre disponible y segura es el principal diferenciador.