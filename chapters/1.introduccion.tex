\chapter{Introducción}
\thispagestyle{fancy}

\justifying 

En el ámbito académico, los estudiantes se apoyan de notas cuando desean aprender, reforzar o poner en práctica alguna habilidad nueva. Tambien, los estudiantes emplean diferentes formas para hacer apuntes. La formas de apuntes son variadas desde el uso de papel y lápiz hasta herramientas digitales. Desde hace 68 años, comenzó el uso de medios electrónicos que permiten un acceso, gestión y toma rápida de notas y apuntes. En la actualidad se ha popularizado el uso de herramientas para la gestión de notas como: Notion, Obsidian, OneNote, entre otras.\\

A lo largo de la vida académica, y profesional, se tiene la necesidad de organizar y reconocer el contenido de grandes volúmenes de documentos, apuntes, y notas. A pesar de la existencia de las herramientas ya mencionadas, estás suelen carecer de funciones que permitan hacer consultas con contexto, ademas dependen de tener una conexión a internet, lo cual dificulta su uso y acceso. Por otro lado, al utilizar estas herramientas, se pierde el control de los recursos generados. Debido a que la información depende de servicios de terceros es importate retomar el control de la informacion. Supongamos dos escenarios, el primero donde las herramientas desaparecen y el segundo donde los proveedores de las herramientas digitales cambian su politica de privacidad e la información, en ambos esenarios la información se pierde. Entonces hay un riesgo porque las herramientas carecen de seguridad y privacidad de la información que se genera por los usuarios.\\


En este contexto, surge la necesidad de desarrollar una aplicación de gestión de notas inteligente que permita a los usuarios crear y consultar una base de conocimientos personalizada, sin depender de una conexión a internet y garantizando la privacidad y seguridad de sus datos. Esta aplicación utilizará tecnologías avanzadas como el procesamiento local del lenguaje natural y sistemas distribuidos para ofrecer una experiencia de usuario eficiente y confiable.
