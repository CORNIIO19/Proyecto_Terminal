\chapter{Propuesta de Solución}

Como se menciono anteriormente una de las limitaciones al hacer uso de plataformas como: Notion, Obsidian, Google Docs o herramientas de Microsoft, entre otras, es que no generan gran confianza a los usuarios, con respecto a temas de privacidad, accesibilidad y además de esto algunas representan un gran reto al conocer y aprender la interfaz con la cual puedan generar notas o información, la curva de aprendizaje puede ser prolongada para llegar a avances y cumplir con un objetivo.\\

Además de satisfacer necesidades de estudiantes y profesionales como la organización del conocimiento y el acceso fácil a través de medios digitales sin importar el espacio físico en el que se encuentren representan un problema a resolver.\\

Por ello, se requiere diseñar una aplicación que genere confianza al usuario, que además busque lograr una conexión cercana entre el conocimiento y el usuario para así lograr ayudar a tener una mejor accesibilidad, mayor confianza y privacidad al conocimiento que cada usuario obtenga.\\

\section{Contexto de uso de la aplicación}
\noindent En esta sección se plantea el escenario o contexto de uso de la aplicación, se presenta desde los problemas que tiene el usuario con sus notas hasta el caso de uso principal que resuelve este problema con la aplicacion Gestor de notas inteligente como se puedo ver en la figura\ref{fig:Contexto}.

\subsubsection{Descripción del contexto de uso}

En la figura\ref{fig:Contexto} podemos observar un flujo que describe el contexto de uso de la aplicacion gestor de notas inteligente, en el cual el principal actor son el usuario y las notas, de esta manera como podemos ver en el \textbf{punto 1} de la figura \ref{fig:Contexto} se muestra como el usuario generá notas y apuntes de diferentes fuentes o de diferentes situaciones en las que unicamente muestra la generacion sin dar una organización o etiquetado pasando en el mismo proceso a la colabroacion de las notas, despues pasamos al \textbf{punto 2} mostramos se muestra el resultado del punto en el que el usuario acumula notas de diversos temas, después en el \textbf{Punto 3} el usuario a partir de estas conjunto de notas que poseé necesita consultarlas o generar nuevas, pero nos encontramos con tres principales problematicas \textbf{Punto 4}, estas probematicas posteriormente nos permitiran presentar la solución de la aplicacion gestor de notas inteligente:
\begin{itemize}
    \item Transporte: El usuario desea consultar sus notas y apuntes mientras hace uso del transporte, esto dificulta el acceso de las notas al no tener un sistema centralizado en el cual recopilarlas.

    \item Organización: El usuario no respeta un orden u organizacion de todas las notas generadas lo cual representa un problema lo cual dificulta su busqueda y retrazan la accion que desee hacer con las notas.

    \item Estudio: El usuario necesita estudiar ayudandose de las notas generadas pero muchas veces solo se necesita estudiar temas en especificos lo cual si el contenido de las notas no esta organizado o bien definido puede retrasar el estudio al dedicar mas tiempo en la busqueda de temas concretos u especificos.
\end{itemize}

\noindent Una vez mostradas estas problematicas podemos ve ren el \textbf{Punto 5}, como estas problematicas causan frustracion al usuario ante la falta de organización, contexto o accebilidad, a partir de esto proponemos el principio de uso de la aplicacion gestor de notas inteligente \textbf{Punto 6} en la que principalmente se propone centralizar las notas que el usuario posea explicandolo de manera en que se tienen el usuario, las notas que el usuario acumule y una aplicacion en la que se puedan centralizar, de esta manera nos da como resultado la creacion de una base de conocimiento, siendo este el punto central de la solución como podemo sver en el \textbf{Punto 7}, una vez centralizadas las notas podemos resolver los problemas de organización, estudio y transporte, generando un sistema clasificador con el cual clasifique las notas relacionando con su contenido encontrando similitudes entre palabras clave o etiquetas \textbf{Punto 8}, dando estructura y orden a las notas del usuario \textbf{Punto 9}, logrando generar un contexto o medio en el cual el usuario podra consultar mediante preguntas o palabras en leguaje natural a la base de conocimiento siendo este el  contexto principal de uso de la aplicacion \textbf{Punto 10}, esto mediante una aplicación \textbf{Punto 12}, en el cual se muestran dos posibles resultados, primero \textbf{Punto 13} en el cual si se encuentra informacion el chatbot nos contestara dando la informacion que encuentre unicamente en la base de conocimeitno del usuario, y la segunda posibilidad \textbf{Punto 14} en la que si el chatbot no encuentra nada informara al usuario y se le dara una recomendacion de que haga alguna nota o adjunte informacion nueva con relación a este tema, dando como resultado el final del contexto de uso de la aplicacion gestor de notas inteligente \textbf{Punto 15} en el que el usuario tendra un mejor acceso a la informacion con consultas totalmente personalizadas con información organizada y lo mas importante con informacion propia.

\begin{figure}[!hptb]
    \centering
    \includegraphics[width=0.99\linewidth]{Contexto_2.png}
    \caption{Contexto de uso de la aplicación}
    \label{fig:Contexto}
\end{figure}

\section{Requerimientos}
\subsection{Funcionales}
\subsection{No Funcionales}


\section{Arquitectura del Sistema}

\section{Prototipos de Interfaz}
