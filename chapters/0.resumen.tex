\chapter*{Resumen}
\thispagestyle{fancy}
\addcontentsline{toc}{chapter}{Resumen}
Este proyecto se creó ante la necesidad de poder crear una base de conocimiento propia a partir de apuntes o notas, de fácil y rápido acceso, en la cual poder tener una mejor organización de los mismos, en la que se pueda mantener y tener conciencia de un orden de esta base de conocimientos. Con el desarrollo de este mismo se plantea como propósito que cada persona que haga uso de esta pueda consultar, estudiar y reforzar el conocimiento generado por sí mismo y demás usuarios a través de notas electrónicas, sin la necesidad de tener que consultar un cuaderno o libro, anteponiendo el uso de la tecnología como medio principal. Para lograr esto, se planea desarrollar el proyecto mediante una metodología ágil basada en Scrum, dividiendo el desarrollo del proyecto en sprint. Con el desarrollo de este proyecto se busca lograr desarrollar una aplicación multiplataforma en la que los usuarios puedan consultar por distintos medios las notas que generen por medios electrónicos, teniendo como beneficio el poder consultar sus notas por medio de un asistente, al que se le puedan generar preguntas que puedan responder con la base de conocimientos que el usuario mismo genere. Este proyecto busca contribuir al orden del conocimiento de los usuarios, dar como beneficio que, sin importar el lugar en el que se encuentren, puedan consultar sus notas y apuntes, buscar la colaboración de los usuarios a través de la generación de bases de conocimientos que sean de acceso de quien lo necesite y tenga interés por aportar o mejorar estas bases.
